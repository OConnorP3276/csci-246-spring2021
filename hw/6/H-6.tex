\documentclass{article}
\usepackage{../fasy-hw}

%% UPDATE these variables:
\renewcommand{\hwnum}{6}
\title{Discrete Structures, Homework \hwnum}
\author{\todo{Patrick O'Connor} (Patrick OConnor (322))}
\collab{n/a}
\date{due: 6 April 2021}

\begin{document}

\maketitle

This homework assignment should be
submitted as a single PDF file both to D2L and to Gradescope.

General homework expectations:
\begin{itemize}
    \item Homework should be typeset using LaTex.
    \item Answers should be in complete sentences and proofread.
    \item You will not plagiarize, nor will you share your written solutions
        with classmates.
    \item List collaborators at the start of each question using the \texttt{collab} command.
    \item Put your answers where the \texttt{todo} command currently is (and
        remove the \texttt{todo}, but not the word \texttt{Answer}).
\end{itemize}


% ============================================
% ============================================
\collab{\todo{}} \nextprob{Applications}
% ============================================
% ============================================

The topics in this class are mathematical in nature, but have strong ties to
computer science: either for understanding how computers work, why alogithms are
correct, or for how to convert real-world data into things that can be analyzed
using a computer.  Explain how each of the following tie to computer science or
data science:

\begin{enumerate}

    \item Tree (be sure that your answer needs a tree, and not a general graph)

        \paragraph{Answer}

        \todo{your answer here}

    \item Directed Graph

        \paragraph{Answer}

        \todo{your answer here}

    \item Equivalence Classes

        \paragraph{Answer}

        \todo{your answer here}

    \item Distance Metrics

        \paragraph{Answer}

        \todo{your answer here}

    \item Distance Metrics

        \paragraph{Answer}

        \todo{your answer here}

    \item Injective Function

        \paragraph{Answer}

        \todo{your answer here}


\end{enumerate}


% ============================================
% ============================================
\collab{\todo{}} \nextprob{Distance Functions}
% ============================================
% ============================================

Suppose you maintain a database of cocktail recipes.  You want to write a
application that asks a user for their favorite cocktail and returns a cocktail
that they might like.  To do so, you define the distance between two cocktails
to be the symmetric distance between their ingredient lists.  Then, for the
input cocktail, you compute the distance to every cocktail in your database and
return one of the cocktails that minimizes this distance.

\begin{enumerate}

    \item Prove that this distance is a metric. (Note: first, formally write out
        what the distance function is, including the domain and codomain).

        \paragraph{Answer}

        \todo{your answer here}

    \item Is this a good distance to use for your application?  Why or why not?

        \paragraph{Answer}

        \todo{your answer here}

\end{enumerate}


% ============================================
% ============================================
\collab{\todo{}} \nextprob{Loop Invariant}
% ============================================
% ============================================

Recall that a loop invariant is used to prove that a loop is correct (which in
turn helps to prove that an algorithm is correct).

\begin{enumerate}

    \item Write pseudocode that uses a while loop to find the second largest
        element in an unsorted array.  The input to your algorithm should be an
        unsorted array $A$ of real values.  The psuedocode should use the
        variable name \textsc{curguess} to store the current guess of the second
        largest value, and \textsc{curmax} to store the current max of $A$.

        \paragraph{Answer}

        \todo{your answer here}

    \item What is the postcondition of the loop? (Call this statement $Q$).

        \paragraph{Answer}

        \todo{your answer here}

    \item What is the precondition of the loop? (Call this statement $P$).

        \paragraph{Answer}

        \todo{your answer here}

    \item What is the loop guard of the loop? (Call this statement $G$).

        \paragraph{Answer}

        \todo{your answer here}

    \item Prove that the loop invariant is when entering the $i^\text{th}$
        iteration of the loop is, ``the variable \textsc{curguess} stores
        the second
        largest value of $A[1,2, \ldots, i]$,
        and \textsc{curmax} stores the largest value of $A[1,2,\ldots, i]$.

        \paragraph{Answer}

        \todo{your answer here}

\end{enumerate}



% ============================================
% ============================================
\collab{\todo{}} \nextprob{Four Colors Suffice}
% ============================================
% ============================================

Read Chapters $9$ and $10$ of \emph{Four Colors Suffice}.

\begin{enumerate}

    \item What is a \emph{reducible configuration}?

        \paragraph{Answer}

        \todo{your answer here}



    \item Who was the first to correctly prove the four color theorem?  What
        role did computers play in the proof?

        \paragraph{Answer}

        \todo{your answer here}




    \item Draw a ``map'' on the M\"obius strip that requires five or more colors
        to color.

        \paragraph{Answer}

        \todo{your answer here}


\end{enumerate}

% ============================================
% ============================================
\collab{\todo{}}
\nextprob{Shafi Goldwasser}
% ============================================
% ============================================

Write a short (1-2 paragraph) biography of Shafi Goldwasser.
\textbf{In your own words}, describe who they are and why they are important in
the history of computer science.

If you use external resources, please provide
proper citations. If you do not use external sources, please write ``I did not
use any sources to write this biography'' as the last sentence of the
biography.

\paragraph{Answer}

\todo{your answer here}

% %% ... the bibliography
% \newpage
% \bibliographystyle{acm}
% \bibliography{biblio}

\end{document}
