\documentclass{article}
\usepackage{../fasy-hw}

%% UPDATE these variables:
\renewcommand{\hwnum}{8}
\title{Discrete Structures, Homework \hwnum}
\author{Patrick OConnor (Patrick OConnor (322))}
\collab{n/a}
\date{due: 30 April 2021}

\begin{document}

\maketitle

This homework assignment should be
submitted as a single PDF file both to D2L and to Gradescope.

General homework expectations:
\begin{itemize}
    \item Homework should be typeset using LaTex.
    \item Answers should be in complete sentences and proofread.
    \item You will not plagiarize, nor will you share your written solutions
        with classmates.
    \item List collaborators at the start of each question using the \texttt{collab} command.
    \item Put your answers where the \texttt{todo} command currently is (and
        remove the \texttt{todo}, but not the word \texttt{Answer}).
\end{itemize}


% ============================================
% ============================================
\collab{n/a} \nextprob{Student Honor Code}
% ============================================
% ============================================

The final exam will be a timed exam.  You can either use LaTex to typeset or you
can scan handwritten answers (if you do the latter, please write legibly and be
sure that your scan is a clear one).  Remember that you must abide by the
student code of conduct.  Specifically, for this exam, you cannot use external
resources (other than your notes, the miro boards, and your submitted HW) or by
exceeding your limited time.  If cheating is detected in this final exam, the
exam will be an automatic F and there may be additional consequences.

Here, please explain why the code of conduct is important for the integrity of
your degree.

\paragraph{Answer}

The code of conduct is important for the integrity of my degree such that if cheating occurs there will 
be repurcussions in the workplace later on as future employers and clients will expect a knowledgeable professional.
Along with this as a student, if I break this code of conduct I am taking away my oppurtunity to learn, develop, and 
improve my skills through this degree. 



% ============================================
% ============================================
\collab{n/a} \nextprob{Probability}
% ============================================
% ============================================

What is the expected value of the following? Justify.

\begin{enumerate}
    \item Face value when selecting a card from a deck.  Assume that the deck of
        cards is a standard 52 card deck, the ordinal cards are worth their face
        value, the face cards (Jacks, Queens, Kings) are
        worth 10 points, and Aces are worth 15 points.
        \paragraph{Answer} Knowing that this is a normal deck we can create a simple equation such that each of the face value 
        has a 4/52 change of being pulled from the deck. And we can create this formula,
        $( \frac{4}{52} )*1+( \frac{1}{6} )*2+( \frac{1}{6} )*3+( \frac{1}{6} )*4+( \frac{1}{6} )*5+( \frac{1}{6} )*6+
        ( \frac{1}{6} )*7+( \frac{1}{6} )*8+( \frac{1}{6} )*9+( \frac{1}{6} )*10+( \frac{1}{6} )*10+( \frac{1}{6} )*10+
        ( \frac{1}{6} )*10+( \frac{1}{6} )*15$

        And with this equation we can conclude that the expected value is roughly 7.692.

    \item Rolling a six-sided die.  Here, the value of a roll is the square of
        the number that appears at the top.  Assume that the die is fairly
        weighted.
        \paragraph{Answer} In order to calculate this, we must simply create a die that has the value squared on it.
        From here it is a simple probability problem and can be approach such that each side has a 1/6
        chance of landing face up. With this we can create our equation,
        $( \frac{1}{6} )*1+( \frac{1}{6} )*4+( \frac{1}{6} )*9+( \frac{1}{6})*16+( \frac{1}{6} )*25+( \frac{1}{6} )*36$
        
        And with this equation we can conclude that the expected value is roughly 15.167.
    
    \item Playing the following game: I flip a fair coin, and you select
        heads/tails while it is in the air.  If you call the correct side,
        you get 10 points.  If not, you roll a die.  If it lands as a
        multiple of three, you get -25 points.
        If it lands on the 5, then you get 100 points.  Otherwise, no points are
        awarded nor deducted.
        \paragraph{Answer} The expected value of points is 7.083. This is can be justified as there are two 
        ways you can initially go in this game. One is a good call which results in you getting 10 points. If a 
        miscall happens you go to a dice roll. This die roll goes two ways as well. One way the die roll can go is if 
        you get a 5. When a 5 is rolled you get 100 points. The other way this can go is if you roll a multiple of 3.
        When this happens, you lose 25 points. Putting this all into a mathematical formula would result in 
    
        ${( \frac{1}{2} )*10 + ( \frac{1}{2} )*( \frac{3}{6} )*-25 + ( \frac{1}{2} )*( \frac{1}{6} )*100}$
       
         and with this the expected value is roughly 7.083 points. 

\end{enumerate}

% ============================================
% ============================================
\collab{n/a} \nextprob{Asymptotic Notation}
% ============================================
% ============================================

Use the definition of big-Theta to prove that $f \colon \R \to \R$ defined by
$f(x) = 3x^2+x$ is $\Theta(x^2)$.

\paragraph{Answer}

With $f(x) = 3x^2+x$ we need to show $c_1 x^2 \leq 3x^2+x \leq c_2 x^2$ for all $x_o \leq x$

We need to find $c_1$ and $c_2$ so we start with $c_1 = 1$

if $c_1 = 1$ then $f(1) = 3^2+1 = 4$.

With that we can take $c_2 = 4$ and $x_o =1$.

Then, $1 x^2 \leq 3x^2+x \leq 4 x^2$ for all $x \geq 1$

Thus, $3x^2+x = O$ $(x^2)$ 


% ============================================
% ============================================
\collab{n/a} \nextprob{Class Participation}
% ============================================
% ============================================

What grade do you think you deserve for class participation? (You can give this
as a letter grade or as a number grade on a 0-100 scale). Please justify by
including approximate number or percent of class periods missed, describing how
you participate in the breakout rooms, and describing any extenuating
circumstances that we should account for.

\paragraph{Answer}

I believe that I deserve a 80 out of 100. I believe this as I did miss a couple classes towards the end of the 
semester as I was forced to start working more than I had anticipated with covid striking some problems in 
my family. Although I missed class I stayed on top of the discord chat, assisting in multiple conversations throughout 
the semester and working through problems with other students when acceptable. For the class periods that I did 
attend, I was attentive and talkative in both lecture and breakout groups. Overall, I made a concious effort to 
be involved in as much of the class as possible based on the extenuating circumstances that I have experienced this 
semester. 


% ============================================
% ============================================
\collab{n/a}
\nextprob{Biography}
% ============================================
% ============================================

Write a short (1-2 paragraph) biography of a mathematician or computer scientist
mentioned in one of the two course texts (that you have not yet written a
biography for yet).
\textbf{In your own words}, describe who they are and why they are important in
the history of computer science.

If you use external resources, please provide
proper citations. If you do not use external sources, please write ``I did not
use any sources to write this biography'' as the last sentence of the
biography.

\paragraph{Answer}

Although we briefly talked about Kenneth Appel in HW-6, we did not write a biography on him and 
only talked about the proof that Wolfgang Haken and him did that was presented in Four Color Suffice.

Kenneth Appel was an american mathematician that was born in Brooklyn, New York in 1932. Growing up Appel recieved a 
bachelors degree from Queens College in 1953. After graduating he was an 
actuary by trade before serving in the United States Army. During his time in the Army, Kenneth was stationed in 
Fort Benning, Georgia and Baumholder, Germany. After two years in the military he finished his doctoral at 
the University of Michigan. After finishing, Appel moved to Princeton, New Jersey with his wife to work for 
the Institute for Defense Analyses. During his time working for the Institute for Defense Analyses, Appel 
researched cryptography. During his later years, Appel joined the Mathematics Department at the University 
of Illinois where he later earned his Ph.D. in 1959. 

The majority of mathematicians have made substantial advancements for theoretical computer science and 
Kenneth Appel is not an excemption. Although Appel did not make any substantial or notable advancements 
in cryptography, I am sure there are some insiders who would say he was an inexpenable worker and think quite 
highly of him based on his mathematical prowess. While cryptography was not his thing, Appel made massive 
advancements in the field of theoretical computer science by introducing and solidying the application of 
computers in proofs. Kenneth Appel and Wolfgang Haken were the first to correctly prove the
four color theorem in 1976. During this time, computers played a crucial role in determining that
four colors would suffice. With the assistance of the work done by German Mathematician
Heinrich Heesch on the concept of reducibility; Appel and Haken were able to determine
an unavoidable set of reducible configurations(1,834 configurations). With this much discussion 
was created on what role computing would play in the world of mathematics and proofs. It is now a respected 
available option that many mathematicians turn to when attempting to calculate or prove complex problems. 
\emph{Britannica}~\cite{britannica}
\emph{Wikipedia}~\cite{wikipedia}


% %% ... the bibliography
 \newpage
 \bibliographystyle{acm}
 \bibliography{biblio}

\end{document}

